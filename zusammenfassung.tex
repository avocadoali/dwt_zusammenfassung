<<<<<<< HEAD
% Created 2022-06-02 Thu 10:24
=======
% Created 2022-06-03 Fri 12:47
>>>>>>> main
% Intended LaTeX compiler: pdflatex
\documentclass[11pt]{article}
\usepackage[utf8]{inputenc}
\usepackage[T1]{fontenc}
\usepackage{graphicx}
\usepackage{longtable}
\usepackage{wrapfig}
\usepackage{rotating}
\usepackage[normalem]{ulem}
\usepackage{amsmath}
\usepackage{amssymb}
\usepackage{capt-of}
\usepackage{hyperref}
\author{Mihir Mahajan, Alfred Nguyen, Noah Kiefer Diaz}
\date{\today}
\title{Zusammenfassung}
\hypersetup{
 pdfauthor={Mihir Mahajan, Alfred Nguyen, Noah Kiefer Diaz},
 pdftitle={Zusammenfassung},
 pdfkeywords={},
 pdfsubject={},
 pdfcreator={Emacs 27.2 (Org mode 9.6)}, 
 pdflang={English}}
\begin{document}

\maketitle
\tableofcontents


\section{Diskrete Wahrscheinlichkeitsräume}
<<<<<<< HEAD
\label{sec:orgd338391}
\subsection{Grundlagen}
\label{sec:orge94c6d7}

\subsubsection{Definition 1}
\label{sec:orgeaa7da6}
=======
\label{sec:org977150b}
\subsection{Grundlagen}
\label{sec:orga74ad3e}

\subsubsection{Definition 1}
\label{sec:orge37f7dd}
>>>>>>> main
\begin{itemize}
\item Ein diskreter Wahrscheinlichkeitsraum ist durch eine \textbf{Ergebnismenge} \(\Omega = \{\omega_1,...,\omega_n\}\) von Elementarereignissen gegeben
\item Jedem Ereignis \$\(\omega\)\textsubscript{i4} ist eine Wahrscheinlichkeit \(0 \leq Pr[\omega_i] \leq 1\) zugeordnet \\
\(\sum_{\omega \in \Omega} Pr[\omega]= 1\)
\item Die Menge \(E \subseteq \Omega\) heißt Ereignis. \(Pr[E] = \sum_{\omega \in E} Pr[\omega]\)
\item \(\bar{E}\) ist komplement zu E
\end{itemize}


Man kann standard Mengenoperationen auf Ereignisse machen, also bei Ereignissen \(A,B\) dann auch \(A \cup B\), \(A \cap B\)

\subsubsection{Lemma 8}
<<<<<<< HEAD
\label{sec:orgc8d1f8f}
=======
\label{sec:org14906e4}
>>>>>>> main
Für Ereignisse \(A,B, A_1, A_2,...,A_n\) gilt
\begin{itemize}
\item \(Pr[\emptyset] = 0, Pr[\Omega] = 1\)
\item \(0 \leq Pr[A] \leq 1\)
\item \(Pr[\bar{A}] = 1 - Pr[A]\)
\item Wenn \(A \subseteq B\) so folgt \(Pr[A] \leq Pr[B]\)
\item Additionssatz: Bei \textbf{paarweise disjunkten} Ereignissen gilt: \\
\(Pr[\bigcup^{n}_{i=1} A_i] = \sum^n_{i=1} Pr[A_i]\) \\
Insbesondere gilt also:\\
\(Pr[A \cup B] = Pr[A] + Pr[B]\) \\
Und für unendliche Menge von \textbf{disjunkten} Ereignissen:\\
\(Pr[\bigcup^{\infty}_{i=1} A_i] = \sum^{\infty}_{i=1} Pr[A_i]\) \\
\end{itemize}

\subsubsection{Satz 9 Siebformel}
<<<<<<< HEAD
\label{sec:org030adde}
Lemma 8, gilt nur für \textbf{disjunkte} Mengen. Das geht auch für nicht disjunkte!
\begin{enumerate}
\item Zwei Mengen
\label{sec:orgba61ad3}
\(Pr[A \cup B] = Pr[A] + Pr[B] - Pr[A \cap B]\)
\item Drei Mengen
\label{sec:org1c3b6e0}
=======
\label{sec:org2f7cd6c}
Lemma 8, gilt nur für \textbf{disjunkte} Mengen. Das geht auch für nicht disjunkte!
\begin{enumerate}
\item Zwei Mengen
\label{sec:org40418f4}
\(Pr[A \cup B] = Pr[A] + Pr[B] - Pr[A \cap B]\)
\item Drei Mengen
\label{sec:org69111b7}
>>>>>>> main
\(Pr[A_1 \cup A_2 \cup A_3] =\) \\
\(Pr[A1] + Pr[A2] + Pr[A3]\) \\
\(- Pr[A1 \cap A2] - Pr[A1 \cap A3] - Pr[A_2 \cap A_3\) \\
\(+ Pr[A_1 \cap A_2 \cap A_3]\)
\item n Mengen
<<<<<<< HEAD
\label{sec:orgba151ce}
=======
\label{sec:org4d4979e}
>>>>>>> main
Veranschaulichen an Venn-Diagramm
\begin{enumerate}
\item Alle aufaddieren
\item Paarweise schnitte subtrahieren
\item Dreifache schnitte dazuaddieren
\item 4- fache schritte subtrahieren
\item \ldots{}
\end{enumerate}
\item für nerds:
<<<<<<< HEAD
\label{sec:orgba0eaf5}
=======
\label{sec:org2d2a9d9}
>>>>>>> main
\(Pr[\bigcup_{i=0}^n  A_i] = \sum_{\emptyset \subset I \subseteq [n]} (-1)^{|I|+1}Pr[\bigcap_{i \in I} A_i]\)
\end{enumerate}

\subsubsection{Wahl der Wahrscheinlichkeiten}
<<<<<<< HEAD
\label{sec:org429ac00}
=======
\label{sec:org83269b5}
>>>>>>> main
Prinzip von Laplace (Pierre Simon Laplace (1749–1827)): Wenn nichts dagegen spricht, gehen wir davon aus, dass alle Elementarereignisse gleich wahrscheinlich sind.
\(Pr[E] = \frac{|E|}{|\Omega|}\)

\subsection{Bedingte Wahrscheinlichkeiten}
<<<<<<< HEAD
\label{sec:org0a5a964}
\subsubsection{Definition 12}
\label{sec:org7a405d0}
=======
\label{sec:org276e34e}
\subsubsection{Definition 12}
\label{sec:org6f4c295}
>>>>>>> main
\(A\) und \(B\) seien Ereignisse mit \(Pr[B] > 0\). Die bedingte Wahrscheinlichkeit \(Pr[A|B]\) von A gegeben B ist definiert als:
\(Pr[A|B] := \frac{Pr[A \cap B]}{Pr[B]}\)

Umgangssprachlich: \(Pr[A|B]\) beschreibt die Wahrscheinlichkeit, dass A eintritt wenn B eintritt.

Die bedingten Wahrscheinlichkeiten \(Pr[·|B]\) bilden für ein beliebiges Ereignis \(B \subseteq \Omega\) mit \(Pr[B] > 0\) einen neuen Wahrscheinlichkeitsraum über \(\Omega\).


\subsubsection{Baba Beispiele}
<<<<<<< HEAD
\label{sec:org97271c7}
\begin{enumerate}
\item {\bfseries\sffamily TODO} Töchterproblem
\label{sec:org8229b3a}
\item {\bfseries\sffamily TODO} Ziegenproblem
\label{sec:org6a4c6f3}
\item {\bfseries\sffamily TODO} Geburtstagsproblem
\label{sec:org43d9c2c}
\end{enumerate}

\subsubsection{Satz 18 (Satz von der totalen Wahrscheinlichkeit)}
\label{sec:org70420d1}
=======
\label{sec:orgbb9b6e5}
\begin{enumerate}
\item {\bfseries\sffamily TODO} Töchterproblem
\label{sec:org7745b05}
\item {\bfseries\sffamily TODO} Ziegenproblem
\label{sec:orgf8cf0a7}
\item {\bfseries\sffamily TODO} Geburtstagsproblem
\label{sec:orgd6865d9}
\end{enumerate}

\subsubsection{Satz 18 (Satz von der totalen Wahrscheinlichkeit)}
\label{sec:orga4739d7}
>>>>>>> main
Die Ereignisse \(A_1, ..., An\) seien paarweise disjunkt und es gelte \(B \subseteq A1 \cup ... \cup An\). \\
\(Pr[B] = \sum_{i=1}^n Pr[B|A_i] * Pr[A_i]\) \\
analog für \(n \rightarrow \infty\)

\subsubsection{Satz 19 (Satz von Bayes)}
<<<<<<< HEAD
\label{sec:org988fd8f}
=======
\label{sec:org31ff9bf}
>>>>>>> main
Es seien \(A_1, ..., A_n\) paarweise disjunkt, mit \(Pr[A_j] > 0\) für alle j.
Außerdem sei \(B \subseteq A_1 \cup ... \cup A_n\) ein Ereignis mit \(Pr[B]>0\).
Dann gilt für beliebiges \(i \in [n]\)

\(Pr[A_i|B] = \frac{Pr[A_i \cap B]}{Pr[B]} = \frac{Pr[B|A_i] * Pr[A_i]}{\sum_{j=1}^n Pr[B|A_j] * Pr[A_j]}\)

\section{Unabhängigkeit}
<<<<<<< HEAD
\label{sec:orgd266003}
=======
\label{sec:orgf950dc7}
>>>>>>> main
Wenn das auftreten von Ereignissen unabhängig ist.
\begin{itemize}
\item \(Pr[A \cap B] = Pr[A] * Pr[B]\)
\item \(Pr[A|B] = Pr[A]\)
\end{itemize}

\section{Zufallsvariablen}
<<<<<<< HEAD
\label{sec:org5a1a35e}

\subsection{Grundlagen}
\label{sec:orgecbfcf2}
=======
\label{sec:orge7fe052}

\subsection{Grundlagen}
\label{sec:orgee8e0af}
>>>>>>> main
Anstatt der Ereignisse selbst sind wir oft an ”Auswirkungen“ oder ”Merkmalen“ der (Elementarereignisse) interessiert

Sei ein Wahrscheinlichkeitsraum auf der Ergebnismenge Ω gegeben. Eine Abbildung \(X : \Omega \rightarrow R\) heißt (numerische) Zufallsvariable.
Eine Zufallsvariable X über einer endlichen oder abzählbar unendlichen Ergebnismenge heißt \textbf{diskret}


\subsection{Erwartungswert und Varianz}
<<<<<<< HEAD
\label{sec:orga776bc8}
\subsubsection{Definition 29}
\label{sec:org2f68425}
=======
\label{sec:orgd206b48}
\subsubsection{Definition 29}
\label{sec:org74a595a}
>>>>>>> main
Zu einer Zufalls variablen \emph{X} definieren wir den \textbf{Erwartungswert} \(E[X]\) durch
\(E[X] := \sum_{x\in W_X} x \ast Pr[X = x] = \sum x \ast f_X(x)\)
sofern \(\sum_{x\in W_X} |x| \ast Pr[X = x]\) konvergiert

\subsubsection{Satz 32 Monotonie des Erwartungswerts}
<<<<<<< HEAD
\label{sec:org83999c8}
=======
\label{sec:orgbea872f}
>>>>>>> main
Seien X und Y Zufallsvariablen über dem Wahrscheinlichkeitsraum \(\omega\) mit \(X(\omega) \leq Y(\omega)\) für alle \(\omega \in \Omega\). Dann gilt \(\mathbb{E}[X] \leq \mathbb{E}[Y]\)
\(\mathbb{E}[X] = \sum_{\omega \in \Omega} X(\omega) * Pr[\omega] \leq \sum_{\omega \in \Omega} Y(\omega) * Pr[\omega] = \mathbb{E}[Y]\)

\subsubsection{Rechenregeln für Erwartungswert}
<<<<<<< HEAD
\label{sec:org6994e77}
=======
\label{sec:orgae9ac87}
>>>>>>> main
Oft betrachtet man eine Zufallsvariable X nicht direkt, sondern wendet noch eine Funktion darauf an:
\(Y := f(X) = f \circ X\) , \\
wobei \(f : D \rightarrow R\) eine beliebige Funktion sei mit \(W_X \subseteq D \subseteq R\).
Beobachtung: \(f(X)\) ist wieder eine Zufallsvariable.

\subsubsection{Satz 33 (Linearität des Erwartungswerts, einfache Version)}
<<<<<<< HEAD
\label{sec:orgf647737}
=======
\label{sec:org532a192}
>>>>>>> main
Für eine beliebige Zufallsvariable \(X\) und \(a, b \in R\) gilt\\
\(\mathbb{E}[a * X + b] = a * \mathbb{E}[X] + b\)

\subsubsection{Satz 34}
<<<<<<< HEAD
\label{sec:orgda7a22c}
=======
\label{sec:org1722f5a}
>>>>>>> main
Sei \(X\) eine Zufallsvariable mit \(W_x \subseteq \mathbb{N}_0\). Dann gilt\\
\(\mathbb{E}[X] = \sum_{i=1}^\infty Pr[X \geq i]\)

\subsubsection{Satz 35}
<<<<<<< HEAD
\label{sec:org5aa6e8f}
=======
\label{sec:org2d68a89}
>>>>>>> main
Sei \(X\) eine Zufallsvariable und A ein Ereignis mit \(Pr[A] > 0\). Die \textbf{bedingte Zufallsvariable} \(X|A\) besitzt die Dichte:\\
\(f_{X|A}(x) := Pr[X=x|A] = \frac{Pr["X=x" \cap A]}{Pr[A]}\) \\
Die Definition ist zulässig, da \\
\(\sum_{x \in W_x} f_{X|A}(x) = \sum_{x \in W_x} \frac{Pr["X=x" \cap A]}{Pr[A]} = 1\) \\
Somit ist \(\mathbb{E}[X|A] = \sum_{x \in W_x} x * f_{X|A}(x)\)

\subsubsection{Satz 36}
<<<<<<< HEAD
\label{sec:orge3342bb}
TODO

\subsubsection{Varianz}
\label{sec:org3b33610}
\begin{enumerate}
\item Definition 38
\label{sec:org1f78d4b}
=======
\label{sec:org6c35eb7}
TODO

\subsubsection{Varianz}
\label{sec:org63b477a}
\begin{enumerate}
\item Definition 38
\label{sec:org029d3fa}
>>>>>>> main
Für eine Zufallsvariable X mit \(\mu = E[X]\) definieren wir die Varianz \(Var[X]\) durch \\
\(Var[X] := E[(X − \mu)^2] = \sum_{x \in W_X}(x − \mu)^2 * Pr[X = x]\) \\
Die Größe \(\sigma := \sqrt{Var[X]}\) \\
Var[X] heißt Standardabweichung von X.
\item Satz 39
<<<<<<< HEAD
\label{sec:org614cc00}
Für eine beliebige Zufallsvariable \(X\) gilt \\
\(Var[X] = \mathbb{E}[X^2] - \mathbb{E}[X]^2\)
\item Satz 41
\label{sec:org7d88ed6}
=======
\label{sec:org9980096}
Für eine beliebige Zufallsvariable \(X\) gilt \\
\(Var[X] = \mathbb{E}[X^2] - \mathbb{E}[X]^2\)
\item Satz 41
\label{sec:org807a0fc}
>>>>>>> main
Für eine beliebige Zufallsvariable \(X\) und \(a,b \in \mathbb{R}\) gilt:\\
\(Var[a*X+b]=a^2*Var[X]\)
\end{enumerate}

\subsection{Mehrere Zufallsvariablen}
<<<<<<< HEAD
\label{sec:org6bb754e}
=======
\label{sec:org69d5936}
>>>>>>> main
Wie kann man mit mehreren Zufallsvariablen über demselben Wahrscheinlichkeitsraum rechnen, auch wenn sie, wie im obigen Beispiel, sehr voneinander abhängig sind?
Wir untersuchen Wahrscheinlichkeiten der Art \\
\(Pr[X = x, Y = y] = Pr[{\omega; X(\omega) = x, Y (\omega)) = y\}]\)

\subsubsection{hypergeometrische Verteilung}
<<<<<<< HEAD
\label{sec:orgad2edc6}
=======
\label{sec:orgb593362}
>>>>>>> main
Allgemein nennt man Zufallsvariablen mit der Dichte
\(Pr[X = x] = \frac{\begin{pmatrix}b \\ x \end{pmatrix} * \begin{pmatrix} a \\ r-x \end{pmatrix}}{\begin{pmatrix}a+b \\ r \end{pmatrix}}\)
\textbf{hypergeometrisch verteilt}. Durch diese Dichte wird ein Experiment modelliert, bei dem r Elemente ohne Zurücklegen aus einer Grundmenge der Mächtigkeit a + b mit b besonders ausgezeichneten Elementen gezogen werden.

\subsubsection{Gemeinsame Dichte}
<<<<<<< HEAD
\label{sec:orgd94be3c}
=======
\label{sec:org74f9b90}
>>>>>>> main
Die Funktion \\
\(f_{X,Y} (x, y) := Pr[X = x, Y = y]\)
heißt gemeinsame Dichte der Zufallsvariablen X und Y. \\
Aus der gemeinsamen Dichte \(f_{X,Y}\) kann man ableiten \\
\(f_X(x) = \sum_{y \in W_Y} f_{X,Y} (x,y)\) \\
\(f_Y(x) = \sum_{x \in W_X} f_{X,Y} (x,y)\) \\
Die Funktionen f\textsubscript{X} und f\textsubscript{Y} nennt man Randdichten. \\
\subsubsection{Unabhängigkeit von Zufallsvariablen}
<<<<<<< HEAD
\label{sec:org6b5551f}
\begin{enumerate}
\item Definition 45
\label{sec:org3f3029e}
=======
\label{sec:org5a2afeb}
\begin{enumerate}
\item Definition 45
\label{sec:orgc15a45c}
>>>>>>> main
Die Zufallsvariablen \(X_1,...,X_n\) heißen unabhängig, wenn für alle \((x_1,...x_n) \in W_{X_1} \times ... \times W_{X_n}\) gilt: \\
\(Pr[X_1 = x_1,...,X_n = x_n] = Pr[X_1 = x_1] * ... * Pr[X_n = x_n]\)

Analog: Gesamte Dichte ist Produkt aus einzelnen Dichten.
Analog: Gesamte Verteilung ist Produkt aus einzelnen Verteilungen.
\end{enumerate}

\subsubsection{Zusammengesetzte Zufallsvariablen}
<<<<<<< HEAD
\label{sec:orgd65918a}
\begin{enumerate}
\item Satz 49
\label{sec:orgfe39b2b}
=======
\label{sec:org575a5d4}
\begin{enumerate}
\item Satz 49
\label{sec:org1f763aa}
>>>>>>> main
Für zwei unabhängige Zufallsvariablen X und Y sei Z := X + Y . Es gilt
\(f_Z(z) = \sum_{x \in W_X} f_X(x) * f_Y(z - x)\)
\end{enumerate}

\subsubsection{Momente zusammengesetzter Zufallsvariablen}
<<<<<<< HEAD
\label{sec:org8eb25e7}
\begin{enumerate}
\item Satz 50 (Linearität des Erwartungswerts)
\label{sec:orge1fd2b8}
=======
\label{sec:org0b3da94}
\begin{enumerate}
\item Satz 50 (Linearität des Erwartungswerts)
\label{sec:org183d557}
>>>>>>> main
Für Zufallsvariablen \(X_1,...,X_n\) und \(X:=a_1X_1 + ... + a_nX_n\) mit \(a_1, ...a_n \in R\) gilt \\
\(\mathbb{E}[X] = a_1 \mathbb{E}[X_1]+ ... + a_n\mathbb{E}[X_n]\)

\item Satz 52 (Multiplikativität des Erwartungswerts)
<<<<<<< HEAD
\label{sec:org21fdf43}
=======
\label{sec:orgca47c82}
>>>>>>> main
Für unabhängige Zufallsvariablen \(X_1,..., X_n\) gilt
\(E[X1*...*Xn] = E[X1]*...*E[Xn]\)

\item Definition 53
<<<<<<< HEAD
\label{sec:org4d1a208}
=======
\label{sec:org90f27a6}
>>>>>>> main
Zu einem Ereignis A heißt die Zufallsvariable \\
\(I_A = \begin{array}{ll}  1 & \, \textrm{falls A eintritt} \\ 0 & \, \textrm{sonst} \\\end{array}\) \\
\textbf{Indikatorvariable} des Ereigniss A
\end{enumerate}

\section{Wichtige diskrete Verteilungen}
<<<<<<< HEAD
\label{sec:org74453ed}

\subsection{Bernoulli Verteilung}
\label{sec:org25e4ce5}
=======
\label{sec:org0c2cf03}

\subsection{Bernoulli Verteilung}
\label{sec:org80c643c}
>>>>>>> main
Eine Zufallsvariable X mit \(W_X = {0, 1}\) und der Dichte
\(f_X(x) = \left\{ \begin{array}{ll} p & x = 1 \\ 1-p & x = 0 \\ \end{array} \right\)

heißt Bernoulli-verteilt. Den Parameter p nennen wir Erfolgswahrscheinlichkeit.
Eine solche Verteilung erhält man z.B. bei einer einzelnen Indikatorvariablen. Es gilt mir \(q := p-1\)
\(E[X] = p\) und \(Var[X] = pq\),
wegen \(E[X^2] = p\) und \(Var[X] = E[X^2] − E[X]^2 = p − p^2\).
\subsection{Binomialverteilung}
<<<<<<< HEAD
\label{sec:orgfedbb69}
Eine Bernoulli-verteilte Zufallsvariable entspricht der Verteilung einer Indikatorvariablen. Häufig betrachtet man jedoch Summen von Indikatorvariablen.
\subsubsection{Definition 55}
\label{sec:org1f33d87}
=======
\label{sec:orgdbbd7d4}
Eine Bernoulli-verteilte Zufallsvariable entspricht der Verteilung einer Indikatorvariablen. Häufig betrachtet man jedoch Summen von Indikatorvariablen.
\subsubsection{Definition 55}
\label{sec:orgf2a51da}
>>>>>>> main
Sei \(X := X_1 + ... + X_n\) als Summe von n unabhängigen, Bernoulli-verteilten Zufallsvariablen mit gleicher Erfolgswahrscheinlichkeit p definiert. Dann heißt X binomialverteilt mit den Parametern n und p. In Zeichen schreiben wir
\(X \sim Bin(n,p)\) \\
\(Pr[X=k] = \begin{pmatrix} n \\ k \end{pmatrix} * p^k * (1-p)^{n-k}\)
\subsubsection{Satz 56}
<<<<<<< HEAD
\label{sec:orgb686d59}
Wenn \(X \sim Bin(n_x, p)\) und \(Y \sim Bin(n_y, p)\) unabhängig sind, dann gilt für \(Z := X + Y\) , dass \(Z \sim Bin(n_x + n_y, p)\).
\subsection{5.3 Geometrische Verteilung}
\label{sec:org3863e00}
Man betrachte ein Experiment, das so lange wiederholt wird, bis Erfolg eintritt. Gelingt ein einzelner Versuch mit Wahrscheinlichkeit p, so ist die Anzahl der Versuche bis zum Erfolg geometrisch verteilt.
\subsubsection{Definition 57}
\label{sec:org620dda6}
=======
\label{sec:org74cdbe3}
Wenn \(X \sim Bin(n_x, p)\) und \(Y \sim Bin(n_y, p)\) unabhängig sind, dann gilt für \(Z := X + Y\) , dass \(Z \sim Bin(n_x + n_y, p)\).
\subsection{5.3 Geometrische Verteilung}
\label{sec:orgeacbd9b}
Man betrachte ein Experiment, das so lange wiederholt wird, bis Erfolg eintritt. Gelingt ein einzelner Versuch mit Wahrscheinlichkeit p, so ist die Anzahl der Versuche bis zum Erfolg geometrisch verteilt.
\subsubsection{Definition 57}
\label{sec:org05d92f4}
>>>>>>> main
Eine geometrisch verteilte Zufallsvariable X mit Parameter (Erfolgswahrscheinlichkeit)
\(p \in (0, 1]\) und \(q := 1 - p\) hat die Dichte
\(f_X(i) = pq^{i-1}\) für \(i \in \mathbb{N}\) .
Für Erwartungswert und Varianz geometrisch verteilter Zufallsvariablen gilt: \\
\(E[X] = \frac{1}{p}\) und \(Var[X] = \frac{q}{p^2}\)
\subsubsection{Warten auf n-ten Erfolg}
<<<<<<< HEAD
\label{sec:orgfebd376}
\(f_Z(z) = \begin{pmatrix} z - 1 \\ n - 1 \end{pmatrix} * p^n(1-p) (1-p)^{z-n}\)
\subsection{Poisson-Verteilung}
\label{sec:orgd58ea2c}
=======
\label{sec:org9c3e16d}
\(f_Z(z) = \begin{pmatrix} z - 1 \\ n - 1 \end{pmatrix} * p^n(1-p) (1-p)^{z-n}\)
\subsection{5.4 Poisson-Verteilung}
\label{sec:org6600172}
>>>>>>> main
Die Poisson-Verteilung kann verwendet werden, um die Anzahl von Ereignissen zu
modellieren, welche mit konstanter Rate und unabhängig voneinander in einem
Zeitintervall auftreten.
Eine Poisson-verteilte Zufallsvariable X mit dem Parameter \(\lambda \geq 0\) hat den
Wertebereich \(W_X = \mathbb{N}_0\) und besitzt die Dichte
\(f_X(i) = \frac{e^{-\lambda}\lambda^i}{i!}\) für \(i \in \mathbb{N}_0\). \\
Als Erwartungswert ergibt sich: \\
\(\mathbb{E}[X] = \lambda\) \\
Und für die Varianz: \\
\(Var[X] = \lambda\)
\subsubsection{5.4.1 Poisson-Verteilung als Grenzwert der Binomialverteilung}
<<<<<<< HEAD
\label{sec:org1b1eece}
=======
\label{sec:org9b85f0e}
>>>>>>> main
Wir betrachten eine Folge von binomialverteilten Zufallsvariablen \(X_n\) mit
\(X_n \sim Bin(n, p_n)\), wobei \(p_n = \lambda/n\). Für ein beliebiges k mit \(0 \leq k \leq n\) ist die Wahrscheinlichkeit, dass \(X_n\) den Wert k annimmt gleich: \\
\(b(k;n,p_n) = \frac{\lambda^k}{k!} * \frac{n^{\underline{k}}}{n^k} (1-\frac{\lambda}{n})^{n-k}\) \\
Damit folgt: \\
\(\lim_{n \rightarrow \infty} b(k;n,p_n) = e^{-\lambda} * \frac{\lambda^k}{k!}\)

\subsubsection{Satz 59}
<<<<<<< HEAD
\label{sec:orgabb5d78}
=======
\label{sec:org8ebd927}
>>>>>>> main
Sind X und Y unabhängige Zufallsvariablen mit \(X \sim Po(\lambda)\) und \(Y \sim Po(\mu)\), dann gilt \\
\(Z := X + Y  \sim Po(\lambda + \mu)\)

\section{Abschätzen von Wahrscheinlichkeiten}
<<<<<<< HEAD
\label{sec:org21ad83b}
\subsubsection{Satz 60 (Markov-Ungleichung)}
\label{sec:orgd3a0a3e}
Sei X eine Zufallsvariable, die nur nicht-negative Werte annimmt. Dann gilt für alle \(t \in R\) mit \(t > 0\), dass \(Pr[X \geq t] \leq E[X]\) \\
Äquivalent dazu: \\
\(Pr[X \geq t * E[X]] \leq 1/t\)
\subsubsection{Satz 61 (Chebyshev-Ungleichung)}
\label{sec:org56ea577}
Sei X eine Zufallsvariable, und sei \(t \in R\) mit \(t > 0\). Dann gilt \\
\(Pr[|X - E[X]| \geq t] \leq \frac{Var[X]}{t^2}\)
äquivalent dazu:
\$Pr[|X - E[X]| \(\ge\) t\sqrt{Var[X]} \(\le\) 1/t\textsuperscript{2}
=======
\label{sec:org0c01e52}
\subsection{Markov \& Chebyshev Ungleichungen}
\label{sec:org8b9fb7b}
\subsection{Satz 60: Markov-Ungleichung}
\label{sec:org8ea2344}
Sei \emph{X} eine Zufallsvariable die nur nicht negative Werte annimmt. Dann gilt für alle \(t\in \mathbb{R}\) mit \(t > 0\), dass
\begin{center}
\(Pr[X>t] \le \frac{\mathbb{E}[X]}{t}\)
\end{center}
Äquivalent dazu: \(Pr[X>t \bullet \mathbb{E}[X]] \le 1/t\)

\subsection{Satz 61: Chebyshev-Ungleichung}
\label{sec:org41e3966}
Sei \emph{X} eine Zufallsvariable, und sei \(t\in \mathbb{R}\) mit \(t>0\). Dann gilt
\begin{center}
\(Pr[|X-\mathbb{E}[X]| \ge t] \le \frac{Var[X]}{t^2}\)
\end{center}
Äquivalent dazu: \(Pr[|X-\mathbb{E}[X]|\ge t\sqrt{Var[X]}] \le 1/t^2\)

\subsection{Satz 62: Gesetz der großen Zahlen}
\label{sec:orgad491e2}
Gegeben sei eine Zufallsvariable \emph{X} \& seien \(\epsilon, \delta > 0\)  beliebig aber fest. Dann gilt für alle \(n\ge \frac{Var[X]}{\epsilon \delta^2}\) \\
Sind \(X_1, ..., X_n\) unabhängige Zufallsvariablen mit derselben Verteilung wie \emph{X} und setzt man \(Z:= \frac{X_1 + ... + X_n}{n}\) \\
so gilt \(Pr[|Z-\mathbb{E}[X]| \ge \delta] \le \epsilon\)

\subsubsection{Wahrscheinlichkeit und relative Häufigkeit}
\label{sec:org9979daf}
\textbf{TODO}: \emph{slide 161}

\subsection{Chernoff-Schranken}
\label{sec:org38a58bd}
>>>>>>> main
\end{document}
