% Created 2022-05-08 Sun 12:51
% Intended LaTeX compiler: pdflatex
\documentclass[11pt]{article}
\usepackage[utf8]{inputenc}
\usepackage[T1]{fontenc}
\usepackage{graphicx}
\usepackage{longtable}
\usepackage{wrapfig}
\usepackage{rotating}
\usepackage[normalem]{ulem}
\usepackage{amsmath}
\usepackage{amssymb}
\usepackage{capt-of}
\usepackage{hyperref}
\author{Mihir Mahajan, Alfred Nguyen, Noah Kiefer Diaz}
\date{\today}
\title{Zusammenfassung}
\hypersetup{
 pdfauthor={Mihir Mahajan, Alfred Nguyen, Noah Kiefer Diaz},
 pdftitle={Zusammenfassung},
 pdfkeywords={},
 pdfsubject={},
 pdfcreator={Emacs 28.1 (Org mode 9.6)}, 
 pdflang={English}}
\begin{document}

\maketitle
\tableofcontents


\section{Diskrete Wahrscheinlichkeitsräume}
\label{sec:org979dd88}
\subsection{Grundlagen}
\label{sec:orgd711c3c}

\subsubsection{Definition 1}
\label{sec:org86e8007}
\begin{itemize}
\item Ein diskreter Wahrscheinlichkeitsraum ist durch eine \textbf{Ergebnismenge} \(\Omega = \{\omega_1,...,\omega_n\}\) von Elementarereignissen gegeben
\item Jedem Ereignis \$\(\omega\)\textsubscript{i4} ist eine Wahrscheinlichkeit \(0 \leq Pr[\omega_i] \leq 1\) zugeordnet \\
\(\sum_{\omega \in \Omega} Pr[\omega]= 1\)
\item Die Menge \(E \subseteq \Omega\) heißt Ereignis. \(Pr[E] = \sum_{\omega \in E} Pr[\omega]\)
\item \(\bar{E}\) ist komplement zu E
\end{itemize}


Man kann standard Mengenoperationen auf Ereignisse machen, also bei Ereignissen \(A,B\) dann auch \(A \cup B\), \(A \cap B\)

\subsubsection{Lemma 8}
\label{sec:org54435dd}
Für Ereignisse \(A,B, A_1, A_2,...,A_n\) gilt
\begin{itemize}
\item \(Pr[\emptyset] = 0, Pr[\Omega] = 1\)
\item \(0 \leq Pr[A] \leq 1\)
\item \(Pr[\bar{A}] = 1 - Pr[A]\)
\item Wenn \(A \subseteq B\) so folgt \(Pr[A] \leq Pr[B]\)
\item Additionssatz: Bei \textbf{paarweise disjunkten} Ereignissen gilt: \\
\(Pr[\bigcup^{n}_{i=1} A_i] = \sum^n_{i=1} Pr[A_i]\) \\
Insbesondere gilt also:\\
\(Pr[A \cup B] = Pr[A] + Pr[B]\) \\
Und für unendliche Menge von \textbf{disjunkten} Ereignissen:\\
\(Pr[\bigcup^{\infty}_{i=1} A_i] = \sum^{\infty}_{i=1} Pr[A_i]\) \\
\end{itemize}

\subsubsection{Satz 9 Siebformel}
\label{sec:orgc15474b}
Lemma 8, gilt nur für \textbf{disjunkte} Mengen. Das geht auch für nicht disjunkte!
\begin{enumerate}
\item Zwei Mengen
\label{sec:org84a20dc}
\(Pr[A \cup B] = Pr[A] + Pr[B] - Pr[A \cap B]\)
\item Drei Mengen
\label{sec:org32ab2b1}
\(Pr[A_1 \cup A_2 \cup A_3] =\) \\
\(Pr[A1] + Pr[A2] + Pr[A3]\) \\
\(- Pr[A1 \cap A2] - Pr[A1 \cap A3] - Pr[A_2 \cap A_3\) \\
\(+ Pr[A_1 \cap A_2 \cap A_3]\)
\item n Mengen
\label{sec:org0d3a096}
Veranschaulichen an Venn-Diagramm
\begin{enumerate}
\item Alle aufaddieren
\item Paarweise schnitte subtrahieren
\item Dreifache schnitte dazuaddieren
\item 4- fache schritte subtrahieren
\item \ldots{}
\end{enumerate}
\end{enumerate}

\subsubsection{Wahl der Wahrscheinlichkeiten}
\label{sec:orga526239}
Prinzip von Laplace (Pierre Simon Laplace (1749–1827)): Wenn nichts dagegen spricht, gehen wir davon aus, dass alle Elementarereignisse gleich wahrscheinlich sind.
\(Pr[E] = \frac{|E|}{|\Omega|}\)

\subsection{Bedingte Wahrscheinlichkeiten}
\label{sec:org0ebc9f6}
\subsubsection{Definition 12}
\label{sec:orgbce5e67}
\(A\) und \(B\) seien Ereignisse mit \(Pr[B] > 0\). Die bedingte Wahrscheinlichkeit \(Pr[A|B]\) von A gegeben B ist definiert als:
\(Pr[A|B] := \frac{Pr[A \cap B]}{Pr[B]}\)

Umgangssprachlich: \(Pr[A|B]\) beschreibt die Wahrscheinlichkeit, dass A eintritt wenn B eintritt.

Die bedingten Wahrscheinlichkeiten \(Pr[·|B]\) bilden für ein beliebiges Ereignis \(B \subseteq \Omega\) mit \(Pr[B] > 0\) einen neuen Wahrscheinlichkeitsraum über \(\Omega\).


\subsubsection{Baba Beispiele}
\label{sec:org6ff1b43}
\begin{enumerate}
\item {\bfseries\sffamily TODO} Töchterproblem
\label{sec:org393affb}
\item {\bfseries\sffamily TODO} Ziegenproblem
\label{sec:org170ca70}
\item {\bfseries\sffamily TODO} Geburtstagsproblem
\label{sec:orgd03dcad}
\end{enumerate}

\subsubsection{Satz 18 (Satz von der totalen Wahrscheinlichkeit)}
\label{sec:orgeb98ea8}
Die Ereignisse \(A_1, ..., An\) seien paarweise disjunkt und es gelte \(B \subseteq A1 \cup ... \cup An\). \\
\(Pr[B] = \sum_{i=1}^n Pr[B|A_i] * Pr[A_i]\) \\
analog für \(n \rightarrow \infty\)

\subsubsection{Satz 19 (Satz von Bayes)}
\label{sec:orgaec011d}
Es seien \(A_1, ..., A_n\) paarweise disjunkt, mit \(Pr[A_j] > 0\) für alle j.
Außerdem sei \(B \subseteq A_1 \cup ... \cup A_n\) ein Ereignis mit \(Pr[B]>0\).
Dann gilt für beliebiges \(i \in [n]\)

\(Pr[A_i|B] = \frac{Pr[A_i \cap B]}{Pr[B]} = \frac{Pr[B|A_i] * Pr[A_i]}{\sum_{j=1}^n Pr[B|A_j] * Pr[A_j]}\)

\section{Unabhängigkeit}
\label{sec:orgb98d953}
Wenn das auftreten von Ereignissen unabhängig ist.
\(Pr[A \cup B] = Pr[A] * Pr[B]\)

\section{Zufallsvariablen}
\label{sec:org153754c}

\subsection{Grundlagen}
\label{sec:orgad9b5cf}
Anstatt der Ereignisse selbst sind wir oft an ”Auswirkungen“ oder ”Merkmalen“ der (Elementarereignisse) interessiert

Sei ein Wahrscheinlichkeitsraum auf der Ergebnismenge Ω gegeben. Eine Abbildung \(X : \Omega \rightarrow R\) heißt (numerische) Zufallsvariable.
Eine Zufallsvariable X über einer endlichen oder abzählbar unendlichen Ergebnismenge heißt \textbf{diskret}
\end{document}
