% Created 2022-05-13 Fri 14:28
% Intended LaTeX compiler: pdflatex
\documentclass[11pt]{article}
\usepackage[utf8]{inputenc}
\usepackage[T1]{fontenc}
\usepackage{graphicx}
\usepackage{longtable}
\usepackage{wrapfig}
\usepackage{rotating}
\usepackage[normalem]{ulem}
\usepackage{amsmath}
\usepackage{amssymb}
\usepackage{capt-of}
\usepackage{hyperref}
\author{Mihir Mahajan, Alfred Nguyen, Noah Kiefer Diaz}
\date{\today}
\title{Zusammenfassung}
\hypersetup{
 pdfauthor={Mihir Mahajan, Alfred Nguyen, Noah Kiefer Diaz},
 pdftitle={Zusammenfassung},
 pdfkeywords={},
 pdfsubject={},
 pdfcreator={Emacs 28.1 (Org mode 9.6)}, 
 pdflang={English}}
\begin{document}

\maketitle
\tableofcontents


\section{Diskrete Wahrscheinlichkeitsräume}
\label{sec:org2057534}
\subsection{Grundlagen}
\label{sec:orgb53598d}

\subsubsection{Definition 1}
\label{sec:org484ae43}
\begin{itemize}
\item Ein diskreter Wahrscheinlichkeitsraum ist durch eine \textbf{Ergebnismenge} \(\Omega = \{\omega_1,...,\omega_n\}\) von Elementarereignissen gegeben
\item Jedem Ereignis \$\(\omega\)\textsubscript{i4} ist eine Wahrscheinlichkeit \(0 \leq Pr[\omega_i] \leq 1\) zugeordnet \\
\(\sum_{\omega \in \Omega} Pr[\omega]= 1\)
\item Die Menge \(E \subseteq \Omega\) heißt Ereignis. \(Pr[E] = \sum_{\omega \in E} Pr[\omega]\)
\item \(\bar{E}\) ist komplement zu E
\end{itemize}


Man kann standard Mengenoperationen auf Ereignisse machen, also bei Ereignissen \(A,B\) dann auch \(A \cup B\), \(A \cap B\)

\subsubsection{Lemma 8}
\label{sec:orgb504dde}
Für Ereignisse \(A,B, A_1, A_2,...,A_n\) gilt
\begin{itemize}
\item \(Pr[\emptyset] = 0, Pr[\Omega] = 1\)
\item \(0 \leq Pr[A] \leq 1\)
\item \(Pr[\bar{A}] = 1 - Pr[A]\)
\item Wenn \(A \subseteq B\) so folgt \(Pr[A] \leq Pr[B]\)
\item Additionssatz: Bei \textbf{paarweise disjunkten} Ereignissen gilt: \\
\(Pr[\bigcup^{n}_{i=1} A_i] = \sum^n_{i=1} Pr[A_i]\) \\
Insbesondere gilt also:\\
\(Pr[A \cup B] = Pr[A] + Pr[B]\) \\
Und für unendliche Menge von \textbf{disjunkten} Ereignissen:\\
\(Pr[\bigcup^{\infty}_{i=1} A_i] = \sum^{\infty}_{i=1} Pr[A_i]\) \\
\end{itemize}

\subsubsection{Satz 9 Siebformel}
\label{sec:org2cb5ae9}
Lemma 8, gilt nur für \textbf{disjunkte} Mengen. Das geht auch für nicht disjunkte!
\begin{enumerate}
\item Zwei Mengen
\label{sec:org73edd7b}
\(Pr[A \cup B] = Pr[A] + Pr[B] - Pr[A \cap B]\)
\item Drei Mengen
\label{sec:org410d987}
\(Pr[A_1 \cup A_2 \cup A_3] =\) \\
\(Pr[A1] + Pr[A2] + Pr[A3]\) \\
\(- Pr[A1 \cap A2] - Pr[A1 \cap A3] - Pr[A_2 \cap A_3\) \\
\(+ Pr[A_1 \cap A_2 \cap A_3]\)
\item n Mengen
\label{sec:org57a9fae}
Veranschaulichen an Venn-Diagramm
\begin{enumerate}
\item Alle aufaddieren
\item Paarweise schnitte subtrahieren
\item Dreifache schnitte dazuaddieren
\item 4- fache schritte subtrahieren
\item \ldots{}
\end{enumerate}
\item für nerds:
\label{sec:org1e461a0}
\(Pr[\bigcup_{i=0}^n  A_i] = \sum_{\emptyset \subset I \subseteq [n]} (-1)^{|I|+1}Pr[\bigcap_{i \in I} A_i]\)
\end{enumerate}

\subsubsection{Wahl der Wahrscheinlichkeiten}
\label{sec:org4fca8cd}
Prinzip von Laplace (Pierre Simon Laplace (1749–1827)): Wenn nichts dagegen spricht, gehen wir davon aus, dass alle Elementarereignisse gleich wahrscheinlich sind.
\(Pr[E] = \frac{|E|}{|\Omega|}\)

\subsection{Bedingte Wahrscheinlichkeiten}
\label{sec:orgaa37907}
\subsubsection{Definition 12}
\label{sec:orgef66845}
\(A\) und \(B\) seien Ereignisse mit \(Pr[B] > 0\). Die bedingte Wahrscheinlichkeit \(Pr[A|B]\) von A gegeben B ist definiert als:
\(Pr[A|B] := \frac{Pr[A \cap B]}{Pr[B]}\)

Umgangssprachlich: \(Pr[A|B]\) beschreibt die Wahrscheinlichkeit, dass A eintritt wenn B eintritt.

Die bedingten Wahrscheinlichkeiten \(Pr[·|B]\) bilden für ein beliebiges Ereignis \(B \subseteq \Omega\) mit \(Pr[B] > 0\) einen neuen Wahrscheinlichkeitsraum über \(\Omega\).


\subsubsection{Baba Beispiele}
\label{sec:org227eddb}
\begin{enumerate}
\item {\bfseries\sffamily TODO} Töchterproblem
\label{sec:org815d3f0}
\item {\bfseries\sffamily TODO} Ziegenproblem
\label{sec:org820140e}
\item {\bfseries\sffamily TODO} Geburtstagsproblem
\label{sec:orgb30a9b7}
\end{enumerate}

\subsubsection{Satz 18 (Satz von der totalen Wahrscheinlichkeit)}
\label{sec:orgf596287}
Die Ereignisse \(A_1, ..., An\) seien paarweise disjunkt und es gelte \(B \subseteq A1 \cup ... \cup An\). \\
\(Pr[B] = \sum_{i=1}^n Pr[B|A_i] * Pr[A_i]\) \\
analog für \(n \rightarrow \infty\)

\subsubsection{Satz 19 (Satz von Bayes)}
\label{sec:org912f268}
Es seien \(A_1, ..., A_n\) paarweise disjunkt, mit \(Pr[A_j] > 0\) für alle j.
Außerdem sei \(B \subseteq A_1 \cup ... \cup A_n\) ein Ereignis mit \(Pr[B]>0\).
Dann gilt für beliebiges \(i \in [n]\)

\(Pr[A_i|B] = \frac{Pr[A_i \cap B]}{Pr[B]} = \frac{Pr[B|A_i] * Pr[A_i]}{\sum_{j=1}^n Pr[B|A_j] * Pr[A_j]}\)

\section{Unabhängigkeit}
\label{sec:orgacd105d}
Wenn das auftreten von Ereignissen unabhängig ist.
\begin{itemize}
\item \(Pr[A \cap B] = Pr[A] * Pr[B]\)
\item \(Pr[A|B] = Pr[A]\)
\end{itemize}

\section{Zufallsvariablen}
\label{sec:org29b3b3d}

\subsection{Grundlagen}
\label{sec:org9d552c1}
Anstatt der Ereignisse selbst sind wir oft an ”Auswirkungen“ oder ”Merkmalen“ der (Elementarereignisse) interessiert

Sei ein Wahrscheinlichkeitsraum auf der Ergebnismenge Ω gegeben. Eine Abbildung \(X : \Omega \rightarrow R\) heißt (numerische) Zufallsvariable.
Eine Zufallsvariable X über einer endlichen oder abzählbar unendlichen Ergebnismenge heißt \textbf{diskret}


\subsection{Erwartungswert und Varianz}
\label{sec:org403cb61}
\subsubsection{Definition 29}
\label{sec:org8bc09f7}
Zu einer Zufalls variablen \emph{X} definieren wir den \textbf{Erwartungswert} \(E[X]\) durch
\(E[X] := \sum_{x\in W_X} x \ast Pr[X = x] = \sum x \ast f_X(x)\)
sofern \(\sum_{x\in W_X} |x| \ast Pr[X = x]\) konvergiert

\subsubsection{Satz 32 Monotonie des Erwartungswerts}
\label{sec:org2f111a5}
Seien X und Y Zufallsvariablen über dem Wahrscheinlichkeitsraum \(\omega\) mit \(X(\omega) \leq Y(\omega)\) für alle \(\omega \in \Omega\). Dann gilt \(\mathbb{E}[X] \leq \mathbb{E}[Y]\)
\(\mathbb{E}[X] = \sum_{\omega \in \Omega} X(\omega) * Pr[\omega] \leq \sum_{\omega \in \Omega} Y(\omega) * Pr[\omega] = \mathbb{E}[Y]\)

\subsubsection{Rechenregeln für Erwartungswert}
\label{sec:org249fedd}
Oft betrachtet man eine Zufallsvariable X nicht direkt, sondern wendet noch eine Funktion darauf an:
\(Y := f(X) = f \circ X\) , \\
wobei \(f : D \rightarrow R\) eine beliebige Funktion sei mit \(W_X \subseteq D \subseteq R\).
Beobachtung: \(f(X)\) ist wieder eine Zufallsvariable.

\subsubsection{Satz 33 (Linearität des Erwartungswerts, einfache Version)}
\label{sec:org2c20180}
Für eine beliebige Zufallsvariable \(X\) und \(a, b \in R\) gilt\\
\(\mathbb{E}[a * X + b] = a * \mathbb{E}[X] + b\)

\subsubsection{Satz 34}
\label{sec:org8776e12}
Sei \(X\) eine Zufallsvariable mit \(W_x \subseteq \mathbb{N}_0\). Dann gilt\\
\(\mathbb{E}[X] = \sum_{i=1}^\infty Pr[X \geq i]\)

\subsubsection{Satz 35}
\label{sec:org9b4c901}
Sei \(X\) eine Zufallsvariable und A ein Ereignis mit \(Pr[A] > 0\). Die \textbf{bedingte Zufallsvariable} \(X|A\) besitzt die Dichte:\\
\(f_{X|A}(x) := Pr[X=x|A] = \frac{Pr["X=x" \cap A]}{Pr[A]}\) \\
Die Definition ist zulässig, da \\
\(\sum_{x \in W_x} f_{X|A}(x) = \sum_{x \in W_x} \frac{Pr["X=x" \cap A]}{Pr[A]} = 1\) \\
Somit ist \(\mathbb{E}[X|A] = \sum_{x \in W_x} x * f_{X|A}(x)\)

\subsubsection{Satz 36}
\label{sec:org1ef20f8}
TODO

\subsubsection{Varianz}
\label{sec:orga7f208f}
\end{document}
